\documentclass{article}
\usepackage[utf8]{inputenc}

\title{CM4 Variance Estimate Explanations}
\author{Paul Chamberlain}
\date{April 2021}

\begin{document}

\maketitle

\section{Introduction}
These are a set of brief notes explaining the methodology of producing variance constrained maps for both the California Current System (CCS) as well as the Gulf of Mexico (GOM). For our purposes, we define the CCS as 20$^\circ$N to 50$^\circ$N latitude and 135$^\circ$W to 105$^\circ$W longitude. The GOM is defined as a region from 20.5$^\circ$N to 30.5$^\circ$N latitude and 100$^\circ$W to 81.5$^\circ$W longitude. 

The CM4 model \cite{held2019structure} is based on a MOM6 based ocean with 1/4$^\circ$ spatial resolution in the horizontal and 75 depth levels in the vertical. The CM4 output has 1$^\circ$ spatial resolution and 35 depth levels. The version of CM4 data used in these calculations was a monthly mean output calculated from 1850 to present day.

CM4 model output that we considered were surface referenced temperature, sea water salinity, ph, oxygen, and chlorophyll. Chlorophyll was only considered for depth levels less than 250 meters. Covariances between variables were only considered for regions deeper than 2062.5 meters. Note that in both the CCS and the GOM, regions of high variability are removed by this definition, so this analysis is inherently biased. Because of memory constraints, the covariance matrix of each set of variables was calculated and saved independently and then reconstructed as a block matrix. 

One of the main innovations that we advance in this analysis is the method through which we localize our results - localization is a technique commonly used in ocean modeling which spatially confines the potential covariance of a grid cell. We do this for the practical purpose of reducing computational load and the scientific purpose of preventing spurious covariances across unphysically long distances. We apply an isotropic localization \cite{gaspari1999construction} with 2 different length scales to the eigenvectors of the covariance matrix. We apply 2 length scales because we assume that the first 4 eigen vectors of the covariance matrix explains variability resulting from large scale forcing or modes, which 


\section{CCS Results}
\section{GOM Results}
\bibliographystyle{ametsoc2014}
\bibliography{references}
\end{document}
